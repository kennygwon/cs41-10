\documentclass[11pt]{article}
\usepackage{fullpage}
\usepackage{clrscode3e}
\usepackage{url}
\usepackage{amssymb} %lots of math symbols
\usepackage{amsmath}
\usepackage{amsthm}
\usepackage{xspace}
\usepackage{graphicx,float}

\DeclareMathOperator{\ptr}{\leq_P}
\DeclareMathOperator{\deq}{:=}

\newtheorem{lemma}{Lemma}

\newcommand{\yes}{\textsc{yes}\xspace}
\newcommand{\no}{\textsc{no}\xspace}
\newcommand{\true}{\textsc{true}\xspace}
\newcommand{\false}{\textsc{false}\xspace}

\newcommand{\prob}[1]{\textsc{#1}\xspace}
\newcommand{\vco}{\prob{VC-Opt}}
\newcommand{\vc}{\prob{Vertex-Cover}}
\newcommand{\vcopt}{\prob{Min-Vertex-Cover}}
\newcommand{\is}{\prob{Independent-Set}}
\newcommand{\isopt}{\prob{Max-Independent-Set}}
\newcommand{\sat}{\prob{Sat}}
\newcommand{\tsat}{\prob{3-Sat}}
\newcommand{\ksat}{\prob{k-Sat}}
\newcommand{\factor}{\prob{Factoring}}
\newcommand{\nfactor}{\prob{Not-Factoring}}
\newcommand{\primes}{\prob{Primes}}
\newcommand{\tdm}{\prob{3d-Matching}}
\newcommand{\tcolor}{\prob{Three-Coloring}}
\newcommand{\wedding}{\prob{Wedding-Planner}}
\newcommand{\clique}{\prob{Clique}}
\newcommand{\cliqueopt}{\prob{Max-Clique}}
\newcommand{\tsp}{\prob{TSP}}
\newcommand{\metrictsp}{\prob{Metric-TSP}}

\newcommand{\np}{\textsc{NP}\xspace}
\newcommand{\npc}{\textsc{NPC}\xspace}
\newcommand{\npcomplete}{\textsc{NP-Complete}\xspace}
\newcommand{\p}{\textsc{P}\xspace}

\newcommand{\titlebox}[6]{
    \begin{center}
        \framebox{
            \vbox{
            \hbox to 5.78in { {\bf #1} \hfill #2}
            \vspace{4mm}
            \hbox to 5.78in { {\Large \hfill #3 \hfill} }
            \vspace{2mm}
            \hbox to 5.78in { {\it Due: #4 \hfill #5} }
        }
    }
    \end{center}
}

\begin{document}
\titlebox{CPSC 41: Algorithms} % course number and title
{your name(s) here} % student(s) name(s) --- this means you (pl)!
{Homework 10} % assignment number
{11:59pm December 6, 2017} % due date
{Professor Bryce Wiedenbeck} % which prof's section

\vspace{0.2in} This homework is due {\bf 11:59pm Wednesday, December 6.}
%
Submit this homework as a {\tt *.tex} file using \textbf{github}.
%
For this homework, you will work with a partner.  It's ok to discuss
approaches with others at a high level, but most of your discussions
should just be with your partner.  The only exception to this rule
is work you've done with a lab partner \emph{while in lab}.  In this
case, note who you've worked with and what parts were solved during
lab.  If there are questions about academic integrity, please visit
the section on Academic Integrity on the course website.


\begin{enumerate}
  \setcounter{enumi}{-1}
\item Before final submission, make sure to fill out the README file.

  \item {\bf (K\&T 11.3)}
    Suppose you are given a set of positive integers $A=\{a_1, a_2,
    \ldots, a_n\}$ and a positive integer $B$.
    A subset $S \subseteq A$ is called \emph{feasible} if the sum of
    the numbers in $S$ does not exceed $B$:
    $$ \sum_{a_i \in S} a_i \leq B.$$
    The sum of the numbers in $S$ will be called the \emph{total sum}
    of $S$.

    You would like to select a feasible subset $S$ of $A$ whose total
    sum is as large as possible.
    
    For example, if $A=\{8,2,4\}$ and $B=11$ then the optimal solution
    is the subset $S=\{8,2\}$.

    \begin{enumerate}
      \item
      Here is an algorithm for this problem.
      \begin{codebox}
        \Procname{\proc{notQuiteRight}($A=\{a_1,\ldots, a_n\}, B$)}
        \li initialize $S = \emptyset$
        \li define $T=0$
        \li \For $i=1$ to $n$: \Do
        \li \If $T + a_i \leq B$ \Then
        \li $S \gets S \cup \{a_i\}$
        \li $T \gets T + a_i$
        \End % if
        \End % for
        \li return $S$
      \end{codebox}
      Give an input for which the total sum of the set $S$ returned by
      this algorithm is less than half the total sum of some other
      feasible subset of $A$. (You don't necessarily have to find the
      optimal subset, just \emph{some} feasible subset.)

    \item Give a polynomial-time approximation algorithm for this
      problem with the following guarantee:
      It returns a feasible set $S \subseteq A$ whose total sum is at
      least half as large as the maximum total sum of any feasible set
      $S' \subseteq A$.
      Your algorithm should run asymptotically faster than $O(n^2)$.
    \end{enumerate}

\item {\bf Reductions.}
  Recall that many problems have a decision version and an
  optimization version, so for example we can consider the problems
  \begin{itemize}
    \item $\is(G,k)$ returns $\yes$ iff there is an independent set in $G$ of
      size $\geq k$,
    \item $\isopt(G)$ returns the size of the largest independent set in
      $G$, 
    \item $\vc(G,k)$ returns $\yes$ iff there is a vertex cover of $G$ of size
      at most $k$, 
    \item $\vcopt(G)$ returns the size of the smallest vertex
      cover of $G$,
    \item $\clique(G,k)$ returns $\yes$ iff there is a clique in $G$
      of size $k$, and
    \item $\cliqueopt(G)$ returns the size of the largest
      clique\footnote{A {\bf clique} is a set of nodes $C \subseteq V$
        such that every two vertices $u, v \in C$ are connected by an
        edge: $\{u,v\} \in E$.} in $G$.
  \end{itemize}

  We know that all \npcomplete problems reduce to each other. It would
  be nice if this meant that an approximation algorithm for one
  \npcomplete problem can be adapted easily into an equally good
  approximation algorithm for any other \npcomplete problem.
  
  \begin{enumerate}
  \item We proved that \vc $\leq_p$ \is by giving an
    algorithm for \vc that used an algorithm for \is as a black box.

    Suppose we wanted to solve the optimization versions of these
    problems, \vcopt and \isopt.
    Can we do the same thing with approximation algorithms?

    That is, suppose we have a black box $2$-approximation algorithm
    for \isopt. 
    Design an approximation algorithm for \vcopt
    using (as a black box) a $2$-approximation algorithm for \isopt.
    (Your algorithm should be based on the reduction.)

    What kind of approximation guarantee can you give?
    Either prove an approximation ratio of your algorithm, or explain
    why this ratio is hard to calculate.

  \item Assume we have an $k$-approximation algorithm for \cliqueopt
    where $k$ is a constant.  Can we use this to
    construct a decent approximation algorithm for \isopt?
    Justify your answer by designing an
    approximation algorithm for \isopt, and either proving an
    approximation ratio or explaining why this ratio is hard to
    calculate.
  \end{enumerate}

\item {\bf Coloring.} 

  Suppose we're somehow told that a graph is
  three-colorable. Could that help us color the graph?  In this
  problem, you'll shoot for a different kind of approximation.  Give a
  polynomial time deterministic algorithm that, given any
  \emph{three-colorable} graph $G = (V,E)$, colors the graph using
  $O(\sqrt{n})$ colors.  Note that the endpoints of each edge
  \emph{must} be different colors, and you're given that it is
  \emph{possible} to color the graph using just three colors, but you
  don't know what the coloring is.

  Here are a few hints to help you along:
  \begin{enumerate}
  \item First, give a simple greedy algorithm that, given a graph $G =
    (V,E)$ such that each vertex has at most $d$ neighbors, colors $G$
    using only $d+1$ colors.
  \item Second, recall the algorithm for deciding if a graph is
    \emph{bipartite}.
  \item Third, start coloring the three-colorable graph taking the
    vertex with the most neighbors, and coloring those neighbors using
    just two colors.
  \end{enumerate}


  \item {\bf Extra credit.}
    Show that the triangle inequality assumption is necessary to
    achieve the $2$-approximation for \metrictsp.  Specifically, show
    that if there is a polynomial-time approximation for general \tsp
    with $\rho(n)=O(1)$, then $\p = \np$.
\end{enumerate}

\end{document}
